\documentclass[12pt,letterpaper]{scrartcl}

% Useful packages
\usepackage{lipsum} % Prints out Lorem Ipsum text
\usepackage[english]{babel} % Handles hyphenation
\usepackage{listings} % Provides commands for generating code listings
\usepackage[margin=1in]{geometry} % Make 1-inch margins
\usepackage{graphicx} % Include images

%% Document Metadata
\title{TSA Airport Screening}
\author{
    Ben Kantor \\
    bdk3079@rit.edu
    \and
    Thomas Moore \\
    tjm3772@rit.edu
    \and
    Christopher Timmons \\
    nerdboy656@yahoo.com
    \and
    Josaphat Valdivia \\
    jxv1308@rit.edu
}
\date{December 3, 2013}

%% End Preamble %%
\begin{document}

\maketitle % Creates the title page (or title section) and displays it in the document

\section{Reflection on the Actor-based design method}
An actor-based design circumvents the problems of shared mutable data (SMD) by avoiding SMD altogether.
Any data that is part of an actor can only be accessed by the Actor from the \texttt{onReceive} method.
A Security check-point was easy enough to create a solution for;
the most difficulties were on the semantics of the Akka framework itself rather than issues arising from concurrency.

\section{Description of changes to the original design}
At the highest level, the design of the solution did not change significantly between the initial submission and the implementation.
There was some initial misunderstanding of the capabilities of Actors and an unknowing reliance on calling methods.
As soon as message-passing was understood as the \emph{only} way to get Actors to communicate with one another, the initial oversights in the design were addressed.

The most notable design challenge was coordinating the end-of-life of the actors.
There are many points at which the Passenger actors in the system would stop executing.
Making sure that this is done correctly meant deciding which Actors were responsible for terminating other Actors at particular stages of the simulation.
Additionally, there were some timing issues which needed to be addressed at the end of the simulated day;
the Actors not only needed to be terminated in a particular order, the termination had to occur as Passengers finished using the system.
This meant that simply sending the \texttt{poisonPill} would be insufficient and would either leave some Passengers un-checked or they would try to send messages to an Actor which had been terminated (resulting in runtime exceptions).

\end{document}
